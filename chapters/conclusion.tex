We have developed GeneNet VR, a Virtual Reality application for the visualization of large biological networks. We used two datasets from MIxT, a real application with visualization problems, as a use case and solved scalability and information overload problems. GeneNet VR gives bioinformaticians the possibility to explore large biological networks for pattern identification using the Oculus Quest, a standalone economical VR headset.

Previous work for the visualization of large biological networks has shown common problems like the hairball problem, cumbersome interactions and scalability issues. We solve these in GeneNet VR by taking advantage of the rich interactivity that VR offers in order to balance out the amount of information. We have implemented several interaction and visualization techniques that help the users explore the large datasets. A locomotion system is used, providing the user a way to move around in the scene, also solving object occlusion problems common in 3D spaces. The user can also move the network around, zoom in and out, and filter the nodes using a -dimensional UI. The edges of the network are shown for each node when the user selects them using a laser pointer. Also, a novel feature allows the user to compare two datasets in simultaneously.

We ran several performance experiments for our case study and demonstrated that GeneNet VR performs well when exploring the networks from MIxT. We also evaluated the performance on the machine for several interactions that are commonly used during the visualization process and obtained an average of 7-8 milliseconds, which is under the 13.9 milliseconds limit that corresponds to 72 FPS required by Oculus. Likewise, we evaluated the performance on the Oculus Quest hardware and the results indicated that the Oculus Quest is around 30\% slower than on the PC, but it still reaches the 72 FPS.

We evaluated the quality of of our project with a qualitative research approach where we conducted purposive sampling with in-depth semi-structured interviews with research scientists from UiT. We obtained very good results, where the respondents highlighted that GeneNet VR is an interesting and useful tool to explore biological networks. The system is also easy to learn for the respondents, even for those new to VR, and the interactivity is smooth, making it easy to explore the networks. We also made a list with further requirements that we extracted from the interviews that will help us improve GeneNet VR in the future.

We believe that GeneNet VR is an important and useful tool for bioinformaticians. We have traced out the guidelines that facilitate VR as an affordable and advantageous option for the visualization of large biological networks. The rich interactivity that VR offers and the advancement in hardware has allowed us to solve visualization and scalability problems that are common in similar visualization tools. The project is open-source and can be accessed here: \url{https://github.com/kolibrid/GeneNet-VR}. We also created a video to show the different interactions that we can do with GeneNet VR to explore large biological networks: \url{https://youtu.be/N4QDZiZqVNY}. Now we ask ourselves, what are the next steps of GeneNet VR? Furthermore, can we use a similar solution to visualize similar networks from other fields? Thanks to the interviews that we carried out, we believe that GeneNet VR has potential in our research problem and can certainly be useful in other fields, such as for the visualization of drug and social networks.
