We have developed GeneNet VR, a Virtual Reality application for the visualization of large biological networks. We used two datasets from MIxT, a real application with visualization problems, as a case study where we solved scalability and information overload problems. GeneNet VR gives bioinformaticians the possibility to explore large biological networks for pattern finding using the Oculus Quest, a standalone cheap VR headset.

Previous work for the visualization of large biological networks has shown common problems like the hairball problem, cumbersome interactions and scalability issues. We solve all of this in GeneNet VR by taking advantage of the rich interactivity that VR offers in order to balance out the amount of information. We have implemented several interaction and visualization techniques that help the users explore these large networks. A locomotion system is used, providing the user a way to move around in the scene, solving also object occlusion problems common in 3D spaces. The user can also move the network around, zoom in it and filter the nodes using a 2-dimensional UI. The edges of the network are shown for each node when the user selects them using a laser pointer. Also, a novel feature allows the users compare two datasets in real time. We demonstrate as well that the networks can be explored in cheap VR hardware by running a performance evaluation, which makes GeneNet VR light to use.

We have evaluated the performance and scalability of GeneNet VR in order to know if it meets the Oculus' performance guidelines and if larger datasets could be visualized. We focused on evaluate the translation and scale of the network as well as the node selection.  We could see that the average delta time oscilates between 6.4 and 8.7 milliseconds in the experiments, which is pretty good. We couldn't determine if the number of nodes and the number of edges to render in the scene could have a big impact in the scalability of the network. A reason for this is that it was hard to isolate these characteristics, since Unity is a complex 3D egine and many processes happen at the same time. In addition we also compared the performance of GeneNet VR in the Oculus Quest Hardware and in the PC hardware. We could see that the performance for the Oculus Quest was slightly worse than on the PC, but still inside of the Oculus' performance guidelines.

All in all, we concluded that using virtual reality for the visualization of abstract networks of data like the ones from MIxT have some benefits. We could see that the interaction in VR can feel more natural for the users because they use their hands. We can also implement some gestures that most people already know (like stretching the network to scale it or pulling from the network to move it around) in order to accelerate the learning process. In addition, VR offers the possibility to add 2D interfaces like menus, which enhances the interactivity and adds multiple interactive possibilities. Some downsides that we we could see in using VR are mostly related to motion sickness and unconformity of using a headset like Oculus Quest.
