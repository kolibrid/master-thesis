% Gene co-expression networks: https://en.wikipedia.org/wiki/Gene_co-expression_network#:~:text=Having%20gene%20expression%20profiles%20of,rise%20and%20fall%20together%20across

It is very cheap nowadays to produce data and many people are doing it due to technological advancement. Just as an example, in the field of genomics, the sequencing of the first human genome (2002) took around 13 years and cost over \$3 million to complete. Now, we can sequence hundreds of genomes in just a few days\cite{big_biological_impacts_bd}. However, the vast amounts of data that we produce can result in problems like data information overload, data interconnectivity and high dimensionality. Therefore, we need better solutions to visualize large amounts of data.

% These problems can be overcome with a visualization system that can scale to large sizes of data and where the interactions are not cumbersome. In this way, we can visualize the data in order to find interesting patterns that will lead to new scientific discoveries
%  \cite{zhang_paciorkowski_craig_cui_2019}.

% Some of the main problems that researchers face when analysing genomic data are: information overload, data interconnectivity and high dimensionality. One way to deal with all this data is to invent novel analysis. However, we still need visual inspection of the data, which is an important challenge, and this is what we attempt to solve. For this reason, it is very important to implement efficient visualization technologies that can lead to find new patterns and the extraction of good conclusions of the data.

In the field of system biology, there are usually network representations where the nodes or bioentities are connected to each other and where these edges represent associations. Networks can increase dramatically in size and complexity and many visualization systems for biological networks have a lack of scalability and the user interactions can be cumbersome. Virtual Reality (VR) has shown to have benefits when visualizing abstract information and offers very rich interactivity solutions \cite{zhang_paciorkowski_craig_cui_2019}. Some specific challenges that we face when visualizing biological data in VR are the following:
\begin{itemize}
  \item Achieve the performance needed for the visualization and interaction with large biological networks.
  \item Understand the scalability limitations for the exploration of biological networks in VR.
  \item VR equipment is expensive and it is also tied to a computer that needs a powerful GPU.
\end{itemize}

% enough computational power to create these large networks, scientific knowledge about large networks a. We need better visualization systems for the analysis and inspection of biological networks and at the same time we need robust applications that can handle the data overload.

[Here I would like to read about how the challenges described in the previous work has been addressed by previous work, and why these solutions have not solved all challenges.]

[What I expect to read here is why your solution is better than the previous solutions described in the previous section.]

We believe that virtual reality (VR) can offer new possibilities for visual inspection in large networks and for the inspection of patterns in these. Even though VR is still a field under exploration, it has been demonstrated that it help scientists work more effectively in fields like medicine \cite{Laver11} \cite{xia_ip_samman_wong_gateno_wang_yeung_kot_tideman_2001} \cite{brain_damage_rehab}, biology \cite{bioinformatics_bti581} \cite{thorley_lawson_duca_shapiro_2008} and neuroscience \cite{bohil_alicea_biocca_2011}\cite{minderer_harvey_donato_moser_2016}, to  cite some examples. VR can be very powerful because it takes advantage of the way the human being perceives and analysis things. We have a great ability to discover patterns; however, we are biologically optimized to see the world and the patterns in 3 dimensions. Some of the advantages that VR has over non-VR approaches are the following:

\begin{enumerate}
  \item View of the environment in 3 dimensions and the possibility to move around the virtual space as in real life, giving a feeling of immersion.
  \item Interaction with the the environment by using hand controllers or the hands themselves, giving the user a feeling of presence.
  \item Possibility to combine 2-dimensional interfaces within the 3-dimensional virtual world that the user can interact with.
\end{enumerate}

We have implemented GeneNet VR, a virtual reality application for the visualization of biological networks. We used two datasets from the MIxT project \cite{dumeaux_fjukstad_interactions_tumor_blood} that contain genetic information from patients with breast cancer. The MIxT project has some a known visual and scalability problems and we have used up-to-date virtual reality interaction techniques to improve it. The techniques that we have used consist on: exploration of the network by moving around the virtual space, making it easier for the user to see the network from different angles; interaction with the network and the nodes to comprehend better the data and the interconnections; and the use of 2-dimensional user interfaces to transform the data.

We used the following methodology during the project: we built an application prototype of GeneNet VR in Unity. Then, we evaluated performance aspects of it in both PC and in the Oculus Quest hardware. Finally, we conducted several interviews with test users where we got feedback about the quality and where we extracted some conclusions and ideas for future work. We concluded from the performance evaluation that GeneNet VR has a good performance for the network sizes that we used. Also, the interactions met the required FPS in general. Only the selection of nodes and showing the relationships showed to have more impact in the performance, but still the application could run smoothly. We also evaluated the performance on the Oculus Quest hardware and concluded that it was also good, meaning that we can visualize these type of networks in relatively affordable headsets like the one we used.

\textbf{Thesis statement: } \emph{Virtual Reality is advantageous for the visualization of large biological networks and for the exploration of patterns in them using affordable hardware}.

\section{Challenges and research problem}

In fields like biology, network visualization seem to be particularly helpful\cite{pujana_network_modeling}\cite{fraser_view_function}. There are many types of relationships that can be measure in a biological context, for example interactions between proteins or genetic interections when revealed by combinations of mutations. All these interactions and correlations can be easier to visualize as a network\cite{merico_visualization}.

MIxT\cite{fjukstad_dumeaux_olsen_lund_hallett_bongo_2017} is a web application for bioinformaticians that was used to identify genes and pathways in the primary tumor that are tightly linked to genes and pathways in the systemic response of a patient with breast cancer\cite{dumeaux_fjukstad_interactions_tumor_blood}. Among other tools, it offers a network visualization of genes which are represented as nodes and edges that represent statistically significant correlation in expression between the nodes.

It can be hard to identify novel patterns when visualizing large amounts of data. Some data structures, like the data networks, also have the challenge of data interconnectivity. These type of data structures represent relationships and are composed by nodes and edges. Even though we have many tools like machine learning, that help researchers automatize and accelerate the pattern recognition process, many times we still need a human expert to inspect these networks\cite{network_expert}.

When exploring a network in MIxT, sometimes it can be difficult to find patterns because of the data overload. This problem happens expecially when there are many nodes and relationships together. In figure \ref{fig:mixt_network} we can see an example of the network visualization from MIxT. As we can see in Figure \ref{fig:mixt_network1}, there are many nodes and relationships; this problem gets worse when we zoom in the network as in Figure \ref{fig:mixt_network_zoom}.

\begin{figure}[h!]
    \centering%
    \begin{subfigure}[t]{0.5\textwidth}
        \centering%
        \includegraphics[width=\linewidth]{mixt_network1}
        \caption{Network with several modules.}
        \label{fig:mixt_network1}
    \end{subfigure}%
    \begin{subfigure}[t]{0.5\textwidth}
        \centering%
        \includegraphics[width=\linewidth]{mixt_network2}
        \caption{Zoom in the network.}
        \label{fig:mixt_network_zoom}
    \end{subfigure}

    \caption{Network view of the MIxT application where nodes repsent genes and the modules are repsented by colors. Relationships are represented by grey lines that connect a gene with another one.}
    \label{fig:mixt_network}
\end{figure}

In Figure \ref{fig:network_biology_evolution} we can see a representation of the evolution for visualization of networks in system biology. Before the computers, networks were represented in 2 dimensions and they were static representations that lacked interactivity, see Figure \ref{fig:network_biology_evolution} A. With the computer era and the advancement in computer graphics, 3D representations were possible, with the addition of interactions (See Figure  \ref{fig:network_biology_evolution} C). As computer science progressed, visualization improved and also new technologies emerged like virtual reality, which has a huge potential with regards to the interactivity (Figure \ref{fig:network_biology_evolution} I).

\begin{figure}[h!]
    \newlength{\tempheight}
    \setlength{\tempheight}{15ex}
    \centering%
    \includegraphics[width=\textwidth]{evolution_visualization}
    \caption{Visualization for network biology. a Undirected unweighted graph showing co-expression relationship between genes. b A 2D representation of a yeast protein-protein interaction network visualized in Cytoscape (left) and potential protein complexes 3D identified by the MCL algorithm from that network (right). c A 3D network of genes showing co-expression relationships. d A multilayered network integrating different types of data visualized by Arena3D. e A hive plot view of a network where nodes are mapped to and positioned on radially distributed linear axes. f Visualization of network changes over time. g Static picture showing part of lung cancer pathway. h Navigation of networks using hand gestures. i Integration and control of 3D networks using VR devices. Figure adapted\cite{pavlopoulos_malliarakis_papanikolaou_theodosiou_enright_iliopoulos_2015}.}
    \label{fig:network_biology_evolution}
\end{figure}%

\section{Proposed solution and contribution}
We have built GeneNet VR, a Virtual Reality visualization system that focuses on solving the problem of visualization of networks with data overload in order to improve the discovery of novel patterns. We used the datasets from the MIxT project so that we work with realistic data sources. We applied Virtual Reality for the visualization and interaction with the networks and offer the following tools that differ from the visualization system built on MIxT:

\begin{itemize}
  \item 3-dimensional view of the network.
  \item Possibility to move around the virtual space to see the network from different angles.
  \item Application of 3-dimensional transformations to the network such as scaling and translation.
  \item Visualization of single node relationships.
  \item Possibility to filter the nodes using a user interface.
  \item Visualization of two networks at the same time in order to compare them in real time.
\end{itemize}

GeneNet VR results in an application that offers a different way for the visualization of biological networks and that gives visualization experts interesting tools to explore the data. The application is developed for Oculus Quest, a cheap standalone VR headset, where we take the advanatage of visualizing the networks without the need of cables and a computer. See Figure \ref{fig:bignet_intro} for an example of GeneNet VR running, where a user explores the blood dataset from MIxT.

\begin{figure}[h!]
    \setlength{\tempheight}{15ex}
    \centering
    \includegraphics[width=\textwidth]{mixt_vr_introduction}
    \caption{A screenshot from GeneNet VR where a user is exploring the blood dataset from MIxT. The node TMED7 is selected and its interconnectionx are shown. The user is filtering the nodes from the dataset using a 2-dimensional menu.}
    \label{fig:bignet_intro}
\end{figure}

For the evaluation of GeneNet VR, we have focused on performance and scalability problems, which we consider important for the experts during the visualization process so that the identification of patterns doesn't get interrupted by slowliness and low FPS. We found out that the performance is good when visualizing datasets similar to the ones from MIxT in both PC and the Oculus Quest headset. As for future work, we would like to test GeneNet VR with larger datasets and with datasets from other fields that are not biology.

\section{Outline}

We have structured the thesis in the following chapters: Chapter 2 descibes how GeneNet VR was implemented, the architecture, design and the Virtual Reality techniques that were used. Chapter 3 focuses on explaining the visualization of MIxT in Virtual Reality. Chapter 4 describes some related projects found in the literature and we compare them with GeneNet VR. In Chapter 5 we describe the experimens and conclusions that carried out in order to evaluate GeneNet VR. In Chapter 6 we explain the conclusions from the project. In Chapter 7 we describe the future work ideas that we have for GeneNet VR.
