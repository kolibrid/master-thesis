\section{Background}
With the increase of biological data, new ways of analysing the produced data have been necessary in order to discover interesting patterns  and make the most out of it. No matter how content-rich or expensively obtained the data is if we don’t obtain anything valuable.

Humans have a great ability to discover patterns, however we are biologically optimized to see the world and the patterns in 3 dimensions. Virtual Reality (VR) is one of the best ways for better discovery in spatial dimensions. It has been demonstrated that VR help scientists work more effectively in fields like medicine \cite{Laver11}\cite{xia_ip_samman_wong_gateno_wang_yeung_kot_tideman_2001} and biology\cite{10.1093/bioinformatics/bti581}, to mention some examples.

\subsection{context}
When the amount of data is considerably big, sometimes it can be difficult to explore it, especially when we are using a 2-dimensional space like a computer screen. One example is the MIxT web application (Matched Interaction Across Tissues)\cite{fjukstad_dumeaux_olsen_lund_hallett_bongo_2017}\cite{dumeaux_fjukstad_interactions_tumor_blood}. MIxT provides a tool to identify genes and pathways in the primary breast tumor that are tightly linked to genes and pathways in the patient blood cells\cite{dumeaux_fjukstad_interactions_tumor_blood}. One of the its features is a network view where genes are represented with nodes and the edges represent statistically significant correlation in expression between the two end-points. However exploring this network and understanding the data and its relationships is hard because there is too much information.
