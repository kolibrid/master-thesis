This section will be dedicated to aspects about VR development and visualization of a gene network in VR. I will cover softwares used for VR development and aspects to take into account in VR like frameworks used, hardware, common problems in VR like locomotion. Talk also about clustering.

\section{VR}

\subsection{Software and frameworks for VR development}
Unity3D\footnote{https://unity.com} and Unreal Engine\footnote{https://www.unrealengine.com} are two popular programs for development of videogames and also virtual reality games and applications. They offer integrations for Oculus Quest and other VR devices in the market. In addition, Oculus Quest offers a devlopment mode that can be activated once the glasses are connected to the PC. In this way the VR application can be tested directly on the VR device.

Virtual Reality development can also be done for the browser. WebVR\footnote{https://webvr.info} is an open specification that makes it possible to experience VR in the browser, no matter what VR device is used. We can find many web frameworks to build VR applications for the web that are based on WebVR. Some of these frameworks are A-frame\footnote{https://aframe.io}, React360\footnote{https://facebook.github.io/react-360} and three.js\footnote{https://threejs.org}.

\subsection{Locomotion and ergonomics}
\begin{itemize}
  \item Physical movement
  \item Script movement
  \item Avatar movement
  \item Steering motion
  \item World pulling
  \item Teleports
\end{itemize}

\subsection{Clustering analysis}
Cluster analysis is used to classify objects or cases into relative groups called clusters. Unlike supervised machine learning techniques, in cluster analysis, there is no prior information about the group or cluster membership for any of the objects. We can find many clustering approaches, two of the most comonly used ones are k-means and DBSCAN.

The k-means algorithm starts by choosing k random centers which can be manually set. Then the data points are assigned to the closest center based on their Euclidean distance.

DBSCAN (Density-Based Spatial Clustering of Applications with Noise) is another algorithm that is based on the density of the data points. The algorithm identifies clusters and expands them by scanning neighborhoods. If it cannot find any points to add, it simply moves on to a new point hoping it will find a new cluster.
