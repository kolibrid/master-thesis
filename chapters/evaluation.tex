% Benchmarking in Unity
% https://blogs.unity3d.com/2018/09/25/performance-benchmarking-in-unity-how-to-get-started/
% Maybe try VRWorks https://developer.nvidia.com/vrworks

% Questions to answer in the evaluation chapter:
% \begin{enumerate}
%   \item{How big can the graph be so that it is comfortable visualizing the network?}\\
%   What is comfortable? Number of FPS?
%   How can we scale the graph? By adding nodes and spread them around, by adding more interconnexions?
%   Should the experiment split in several parts? Scaling, filtering, moving around, etc.
%   What is the performance by using Oculus Link and the performance using just the Quest hardware?
%   -We can use the Unity GPU Profiler for Oculus Quest and Go in order to see the performance.\\
%   \href{https://developer.oculus.com/blog/getting-started-w-the-unity-gpu-profiler-for-oculus-quest-and-go/}{See: Getting Started w/ The Unity GPU Profiler for Oculus Quest and Go}
%   \item{How is this way of visualizing the graph better by using VR?}\\
%   We are researchging the technology and the test with actual users is for future work.
%   \item{In what way can the application and the visualization of the graph be improved?}\\
%   Argue in the discussion part.
% \end{enumerate}

% Links
% Profiler panel

BigNet VR has been implemented to explore biological networks like the one from MIxT. In this chapter we will evaluate the scalability of the application for this type of networks. The following list shows the questions that were asked as part of the evaluation and that we will try to answer along this chapter:
\begin{enumerate}
  \item How big can the biological network be so that it is comfortable to explore and visualize it?
  \item How can we scale the network?
  \item What is the performance by using Oculus Link and the performance using just the Quest hardware?
  \item How comfortable is it to explore a network?
\end{enumerate}

\section{Experimental setup}
% @TODO Write about the specs of the system being used for the evaluation: the machine, software, etc.
We ran the experiments in a machine with Windows 10 and the following hardware specification:
\begin{itemize}
  \item Processor: Intel(R) Xeon(R) CPU E3-1275 v6 @ 2.80GHz 3.79 GHz.
  \item RAM memory: 64.0 GB.
  \item System type: 64-bit Operating System.
\end{itemize}

GPU specifications:
\begin{itemize}
  \item Adapter type: NVIDIA GeForce GTX 1080 Ti.
  \item Chip Type: GeForce GTX 1080 Ti.
  \item DAC Type: Integrated RAMDAC.
  \item Total Available Graphics Memory: 45025 MB.
\end{itemize}

% @TODO Write about Oculus quest hardware
Oculus Quest specifications
\begin{itemize}
  \item Panel Type: Dual OLED 1600x1440.
  \item Supported Refresh Rate: 72Hz.
  \item Tracking: Inside out, 6DOF.
  \item CPU: Qualcomm® Snapdragon 835.
  \item GPU: Qualcomm® Adreno™ 540 GPU.
  \item Memory: 4GB total.
\end{itemize}

\section{Performance of the system}
% @TODO Write about how scalability was tested.
A test scene was built in order to test the network and the scalability of it. A random network is built and we test how comfortable it is to navigate through it. This is measured by the FPS rate.

VR profiling is a technique used to get an overview of the performance of our application. This is usually done in order to find bottlenecks so that we can eliminate them and improve the application's performance.

According to Oculus' performance baselines\cite{oculus_performance_baselines}, an application should meet the following requirements:
\begin{itemize}
  \item 72 FPS for Oculus Quest (required by Oculus).
  \item 50-100 draw calls per frame.
  \item 50,000-100,000 triangles or vertices per frame.
\end{itemize}

To profile the application we used the built-in profiler in Unity, the software used for the development. The Unity Profiler gives information about per-fram CPU and GPU performance metrics.

\section{Scalability of the system}

\section{Questionnaire to evaluate the system}
One of the questions that we asked ourselves during the evaluation process was about the comfortability of using BigNext VR to explore a biological network. This is an important aspect when building VR applications. Some of the aspects to take into account are for instance the motion sickness or the intuitiveness. In order to evaluate this we made a questionnaire for bioinformaticians that would test the application. Unfortunately, due to the Covid-19 situtation\cite{covid_19}, we were not able to run this testing process because it was not possible to have people testing the application on a single Oculus Quest device without avoiding the social distancing rules.
We estimated to have tested the project with at least 10 participants with knowdledge in bioinformatics. With this number of participants we can make some statistics. The questions that we prepared for the questionnaire were the following:\\

Indicate the level of agreement or disagreement with each of the following statements or just mark yes or no:
\begin{enumerate}
  \item Have you ever used a VR headset before?\\
  Yes / No

  \item I feel comfortable using the Oculus Quest headset.\\
  Strongly agree / Agree / Neutral / Disagree / Strongly Disagree

  \item I feel comfortable moving around the virtual environment using the teleport functionality.\\
  Strongly agree / Agree / Neutral / Disagree / Strongly Disagree

  \item I feel comfortable rotating to any direction.\\
  Strongly agree / Agree / Neutral / Disagree / Strongly Disagree

  \item I feel comfortable visualizing the network by moving my head.\\
  Strongly agree / Agree / Neutral / Disagree / Strongly Disagree

  \item I feel comfortable selecting the nodes to visualize the relationships.\\
  Strongly agree / Agree / Neutral / Disagree / Strongly Disagree

  \item I feel comfortable moving the network to the position that I want.\\
  Strongly agree / Agree / Neutral / Disagree / Strongly Disagree

  \item I feel comfortable scaling the network.\\
  Strongly agree / Agree / Neutral / Disagree / Strongly Disagree

  \item I feel comfortable opening the UI menu to filter the data from the network.\\
  Strongly agree / Agree / Neutral / Disagree / Strongly Disagree

  \item It is intuitive to manipulate the network using the controllers.\\
  Strongly agree / Agree / Neutral / Disagree / Strongly Disagree

  \item The different actions in the controllers are easy to learn and remember.\\
  Strongly agree / Agree / Neutral / Disagree / Strongly Disagree

  \item I can move the network to any position that I want.\\
  Strongly agree / Agree / Neutral / Disagree / Strongly Disagree

  \item I can scale the network to any size that I want.\\
  Strongly agree / Agree / Neutral / Disagree / Strongly Disagree

  \item I can select any node that I want.\\
  Strongly agree / Agree / Neutral / Disagree / Strongly Disagree

  \item I can easily visualize the relationships of any node.\\
  Strongly agree / Agree / Neutral / Disagree / Strongly Disagree

  \item I can easily filter the data in the network that I want to visualize.\\
  Strongly agree / Agree / Neutral / Disagree / Strongly Disagree
\end{enumerate}
