\documentclass[USenglish]{uit-thesis}

\usepackage[backend=bibtex,style=numeric,sorting=none]{biblatex}
\usepackage{gensymb}
\usepackage{pgfplots}
\usepackage{pgfplotstable}
\usepackage{tikz}
\usepackage[toc,page]{appendix}
\usepackage{adjustbox}
\usepackage{tabularx}
\bibliography{main}
\usepackage[utf8]{inputenc}

\graphicspath{{images/}}

% Define colours
\definecolor{bblue}{HTML}{72A9DA}
\definecolor{rred}{HTML}{CA422D}
\definecolor{ggreen}{HTML}{95C380}

\begin{document}

%:-------------------------- Frontpage ------------------------

\title{GeneNet VR: Large Biological Networks in Virtual Reality Using Inexpensive Hardware}
% \subtitle{Subtitle}% Note: this is optional, and may be commented out
\author{\'Alvaro Mart\'inez Fern\'andez}
\thesisfaculty{Faculty of Science and Technology \\ Department of Computer Science}
\thesisprogramme{INF-3990 Master's thesis in Computer Science November 2020}

\pgfplotsset{width=\textwidth}

\maketitle

%:-------------------------- Frontmatter -----------------------
\frontmatter

%: Commented out
% \iffalse
% \begin{dedication}
% To...
%
% Thanks for...
% \end{dedication}

\begin{epigraph}
\epigraphitem{Anything created by human beings is already in the great book of nature.}{Antoni Gaudi}
\epigraphitem{There are decades of innovations ahead. We’re at the very beginning, where it’s just at the stage where we can bring in consumers [but] there’s so much further to go from there.}{Brendan Iribe}
\end{epigraph}

\begin{abstract}
Biological data is often visualized as networks. However, they face problems such as information overload, high interconnectivity and high dimensionality. Existing approaches try to solve these problems by reducing the interactivity in favor of more information or by using specific hardware. This thesis aims to solve these problems using Virtual Reality (VR) and the Oculus Quest, an affordable VR headset. Thanks to the rich interactivity that VR offers and the advancements in hardware, we can reduce the amount of information in favor of more interactivity.

In order to test our hypothesis that Virtual Reality can be advantageous in the visualization of large biological networks, we built GeneNet VR, a prototype of a VR application for the Oculus Quest. As a case study, we used two gene networks from MIxT, a real application with visualization issues. We evaluated the performance and scalability of GeneNet VR and we conducted in-depth semi-structured interviews with several research scientists to evaluate the quality of the application.

The results showed that the performance of GeneNet VR for the datasets from our use case reaches the 72 FPS required by the Oculus' performance guidelines and that GeneNet VR scales well for the largest dataset with 2693 nodes. We also evaluated the performance of GeneNet VR on the Oculus Quest hardware, which also reached the established limit of 72 FPS from Oculus. The Oculus Quest is therefore an affordable option for the visualization of large datasets. From the interviews, we also learned that GeneNet VR has potential as a visualization tool for large biological networks and that is easy to use even for novice VR users. Thus, VR hardware like the Oculus Quest should be considered a competitive solution for visualization tools, as described in this thesis.

\end{abstract}

\begin{acknowledgement}

I would first like to thank my advisor, Associate Professor Edvard Pedersen, for his guidance through this thesis, his constant encouragement, and for the interesting discussions about Virtual Reality and Computer Science. I would also like to thank my co-advisor, Lars Ailo Bongo, for his great knowledge in Bioinformatics, his good criticism, and for all the great insights about scientific writing. Thank you also Vanessa Dumeaux, for introducing me to MIxT and for your great contributions to Biology.

I would also like to thank the collaboration of the several research scientists that participated in the evaluation of my project and that contributed to their knowledge in Biology and Computer Science.

To my close friends Juncal García García, Mireia Nager and Reidar Staupe-Delgado for all the great experiences in Tromsø and for teaching me the value of science.

To Ramsalt Lab, for making it possible for me to study and expand my knowledge in Computer Science.

I would like to thank also my family, for their support and encouragement even from the distance.

Finally, thank you Dominic Ochotorena for your love, for always supporting me, and for being there.

\end{acknowledgement}
% \fi
%: End comments

\tableofcontents

%:-------------------------- Mainmatter -----------------------
\mainmatter

\chapter{Introduction}
\section{Background}
With the increase of biological data, new ways of analysing the produced data have been necessary in order to discover interesting patterns  and make the most out of it. No matter how content-rich or expensively obtained the data is if we don’t obtain anything valuable.

Humans have a great ability to discover patterns, however we are biologically optimized to see the world and the patterns in 3 dimensions. Virtual Reality (VR) is one of the best ways for better discovery in spatial dimensions. It has been demonstrated that VR help scientists work more effectively in fields like medicine \cite{Laver11}\cite{xia_ip_samman_wong_gateno_wang_yeung_kot_tideman_2001} and biology\cite{10.1093/bioinformatics/bti581}, to mention some examples.

\subsection{Context}
When the amount of data is considerably big, sometimes it can be difficult to explore it, especially when we are using a 2-dimensional space like a computer screen. One example is the MIxT web application (Matched Interaction Across Tissues)\cite{fjukstad_dumeaux_olsen_lund_hallett_bongo_2017}\cite{dumeaux_fjukstad_interactions_tumor_blood}. MIxT provides a tool to identify genes and pathways in the primary breast tumor that are tightly linked to genes and pathways in the patient blood cells\cite{dumeaux_fjukstad_interactions_tumor_blood}. One of the its features is a network view where genes are represented with nodes and the edges represent statistically significant correlation in expression between the two end-points. However exploring this network and understanding the data and its relationships is hard because there is too much information.

\begin{figure}[h!]
    \newlength{\tempheight}
    \setlength{\tempheight}{15ex}
    \centering%
    \begin{subfigure}[t]{0.5\textwidth}
        \centering%
        \includegraphics[width=\linewidth]{mixt_network1}
        \caption{Network with several modules.}
        \label{fig:mixt_network1}
    \end{subfigure}%
    \begin{subfigure}[t]{0.5\textwidth}
        \centering%
        \includegraphics[width=\linewidth]{mixt_network2}
        \caption{Zoom in the network.}
        \label{fig:mixt_network_zoom}
    \end{subfigure}

    \caption{Network view of the MIxT application where nodes repsent genes and the modules are repsented by colors. Relationships are represented by grey lines that connect a gene with another one.}
    \label{fig:mixt_network}
\end{figure}

Figure \ref{fig:mixt_network} shows an example of the network visualization from MIxT. As we can see in Figure \ref{fig:mixt_network1}, there are many nodes and relationships among them and when we zoom in in the network, it becomes too difficult to understandt he data and the relationships as shown in in Figure \ref{fig:mixt_network_zoom}.

\subsection{Software and frameworks for VR development}
Unity3D\footnote{https://unity.com} and Unreal Engine\footnote{https://www.unrealengine.com} are 2 of the most used tools for development of virtual reality applications. They offer integrations for Oculus Quest and other VR devices in the market.In addition Oculus quest offers a devlopment mode which can be activated once the glasses are connected to the PC. In this way the VR application can be tested directly on the VR device.

We can also find some web frameworks to build VR applications for the web. One of the most popular are A-frame\footnote{https://aframe.io/} and React360\footnote{https://facebook.github.io/react-360/}.


\chapter{GeneNet VR}
\input{chapters/genenetvr}

\chapter{MIxT}
% What is MIxT used for? And how would users do it in VR?
% How the MIxT application is implemented in GeneNet VR and what it does.
% What are typical network characteristics for this (type of) application? Size, clusters, edges, connectivity, etc?

The Matched Interaction Across Tissues (MIxT) is a system developed by UiT and Concordia University for exploring and comparing transcriptional profiles from two or more matched tissues across individuals \cite{fjukstad_dumeaux_olsen_lund_hallett_bongo_2017}. This system is implemented as a web application and it has a 2-dimensional visualization tool to explore biological networks \footnote{https://mixt-tumor-stroma.bci.mcgill.ca/network}. This tool has, however, some known scalability and visualization problems.

We have used MIxT in GeneNet VR as a case study. In addition, we used this case study in the evaluation since it is a realistic application where we used complex networks that originally didn't scale in the existing web application. In this chapter we will describe what MIxT is used for, its disadvantages for scaling large biological networks and the challenges of building a VR visualization system that can solve the visualization and scalability problems.

\section{What is MIxT used for?}
MIxT is a web application for interactive data exploration in system biology developed by UiT and Concordia University. A reasearch was carried out for the study of interactions between the tumor and the blood systemic response of breast cancer patients. In the study, they profiled RNA in blood and matched tumor from 173 patients with breast cancer. The goal of the study was to identify genes and pathways in the primary tumor that are tightly linked to genes and pathways in the patient's systemic response (SR). The SR is the body's response to an infectious or non infectious insult. A biological pathway is a series of actions among the molecules in a cell that leads to a certain product or change in the cell. The result of the study suggests new ways of monitor breast cancer by looking outside the tumor and studying the patient's systemic response.

MIxT provides an interactive view for networks. In Figure \ref{fig:mixt_webapp}, we show a screenshot from the network view. This view is two-dimensional and the user can do some interactions to explore data, but these are very limited. The users can zoom in and zoom out and also drag and drop with the mouse in the view in order to "move around". Also, by hovering on a node, the user can see the name of that node. By clicking on a node, the user goes to another page with more detailed biological information about that gene.

\begin{figure}[h!]
    \setlength{\tempheight}{15ex}
    \centering
    \includegraphics[width=\textwidth]{mixt_webapp}
    \caption{Screenshot from the network view in the MIxT web application.}
    \label{fig:mixt_webapp}
\end{figure}

An evident problem in this view appears when we start to explore the network. The clusters have many nodes and each node can have multiple edges, so the graph ends up looking like a hairball and it is not easy to explore (see Figure \ref{fig:mixt_hairball} for a screenshot of this problem from MIxT). In addition, the edges need to be rendered every time the user interacts with the network. It may take a few secons until the user can view all the edges that are in the view frame. This is a problem since the user needs to do many interactions when exploring the network. The visualization process becomes cumbersome and tedious in MIxT.

\begin{figure}[h!]
    \setlength{\tempheight}{15ex}
    \centering
    \includegraphics[width=\textwidth]{mixt_hairball}
    \caption{Hairball problem in the network view in MIxT.}
    \label{fig:mixt_hairball}
\end{figure}

\section{MIxT in VR}
We have implemented a Virtual Reality version of the network view from MIxT in order to solve some scalability and visualization problems that this tool has. In addition to the originally challenges that the network visualizer has, there other challenges that we need to take into account in VR. In this section, we are going to explain these challenges and how we solved them in GeneNet VR.

When moving the original visualization system to VR, we have to count with a new dimension and also the immersive feeling for the user. Having a new dimension is an advantage when we want to visualize high-dimensional data, like gene networks. However, we need to cope with occlusion problems. When the user visualizes the network from a particular angle, the nodes and edges that are in front of the user's viewpoint may hide other nodes and edges that are behind them. To solve this problem, we need to give the user the possibility to view the network from different angles. In GeneNet VR we have implemented a locomotion solution that allows the user to teleport to other parts of the scene. The user can also rotate the viewpoint to the right or to the left. In addition, the user has the possibility to move the network around by using the VR right controller.

As we mentioned before, the original network view from MIxT has an information overload problem. The network looks like a hairball and it's hard to explore it and find patterns in it. To solve this problem in VR, we show only the edges from the nodes that the user wants to explore. This solution reduces the amount of information that is shown, but it needs some interaction solutions so that the user can easily explore the edges of the network. In GeneNet VR, we have implemented a node selector that allows the user to select a particular node using a laser pointer. When the laser collides with a node, the edges of this node are shown. Also, the user can scale up and down the network by using the VR controllers. In addition, a filtering menu was implemented in GeneNet VR, where the user can filter out the information from the network that is not relevant.

Another problem from the MIxT network view is that it can be slow when visualizing the data. The nodes may take a few seconds to show completely, so the visualization process can be hard.

\section{Network characteristics}
The human being has 23 pairs of chromosomes in each of our cells. Each chromosome can contain from hundreds to thousands of genes. There are estimated to be around 30.000 genes. In GeneNet VR, the blood dataset is the biggest one, and contains a total of 2693 genes. We don't expect to deal with a network of 30.000, but it is true, as we can see from the datasets, that they can contain several thousands nodes and also several thousand edges. For instance, in the blood dataset, we can find a node that has 1607 relationships to other nodes.

[What are typical network characteristics for this (type of) application? Size, clusters, edges, connectivity, etc?]


\chapter{Evaluation and discussion}
% Benchmarking in Unity
% https://blogs.unity3d.com/2018/09/25/performance-benchmarking-in-unity-how-to-get-started/
% Maybe try VRWorks https://developer.nvidia.com/vrworks

% Questions to answer in the evaluation chapter:
% \begin{enumerate}
%   \item{How big can the graph be so that it is comfortable visualizing the network?}\\
%   What is comfortable? Number of FPS?
%   How can we scale the graph? By adding nodes and spread them around, by adding more interconnexions?
%   Should the experiment split in several parts? Scaling, filtering, moving around, etc.
%   What is the performance by using Oculus Link and the performance using just the Quest hardware?
%   -We can use the Unity GPU Profiler for Oculus Quest and Go in order to see the performance.\\
%   \href{https://developer.oculus.com/blog/getting-started-w-the-unity-gpu-profiler-for-oculus-quest-and-go/}{See: Getting Started w/ The Unity GPU Profiler for Oculus Quest and Go}
%   \item{How is this way of visualizing the graph better by using VR?}\\
%   We are researchging the technology and the test with actual users is for future work.
%   \item{In what way can the application and the visualization of the graph be improved?}\\
%   Argue in the discussion part.
% \end{enumerate}

% Links
% Profiler panel

BigNet VR has been built to explore biological networks that contain genetic information from human beings. We have used two  networks from MIxT where we applied several VR techniques to build a visualization system. In this chapter we want to evaluate this system focusing on scalability and performance problems. Since we are visualizing networks with genetic information using VR, we want to know if the system that we built can be used for any size of this type of networks. As part of the evaluation process, we designed a list of questions that we will try to answer along this chapter. These questions are:
\begin{enumerate}
  \item For which interactions do we achieve the recommended FPS (72) when scaling the network?
  \item What characteristics of the network influence the scalability?
  \item What is the performance by using Oculus Link and the performance using just the Quest hardware?
  % \item Bonus: how will “beautifications” influence scalability?
  \item How do users perceive the interaction of the network?
\end{enumerate}

Question one is based on the Oculus' performance guidelines\cite{oculus_performance_baselines}, that say that an application should meet the following.
\begin{itemize}
  \item 72 FPS for Oculus Quest (required by Oculus).
  \item 50-100 draw calls per frame.
  \item 50,000-100,000 triangles or vertices per frame.
\end{itemize}

As for the second question, we want to know what characteristics influence the scalability of the network and the visualization system. Some elements of the network have more impact in the performance than others. To keep it simple, we will evaluate the impact in the performance done by the following elements, that are the most important for the visualization: number of nodes, number of lines and number of clusters.

We will also evaluate the performance of the application being run in the Oculus Quest hardware and also in a PC, using the Oculus Link cable that connects the headset to the machine. The hardware of the Oculus Quest is not as powerful as the one of the machine that was used for the development. We would like to know if the performance is good in both the machine and the headset.

Finally, as the last question says, we want to know how the users perceive the interaction and visualization of the network. We will evaluate this with a qualitative method with a demo of the application using the MIxT network and also an interview.

\section{Experimentation plan}
% @TODO Write about the specs of the system being used for the evaluation: the machine, software, etc.
An experimentation plan was designed to ensure that the experiments are consistent and that they can be repeated several times getting realistic measurements. We take into consideration the following aspects for our experiments:
\begin{enumerate}
  \item Scalability for different interactions.
  \item Network characteristics.
  \item Bottlenecks.
  \item User study.
  \item Hardware and software specification.
\end{enumerate}

There are some elements of the network that influence the scalability of it, like the number of nodes,

We ran the experiments in a machine with Windows 10. In Table \ref{tab:machine-specs} we can see some of the hardware specification from the machine used:
\begin{table}[h!]
\centering
\begin{tabular}{ll}
\multicolumn{2}{c}{Machine specification}                        \\
Processor   & Intel(R) Xeon(R) CPU E3-1275 v6 @ 2.80GHz 3.79 GHz \\
RAM memory  & 64.0 GB                                            \\
System type & 64-bit Operating System
\end{tabular}
\caption{Machine specification.}
\label{tab:machine-specs}
\end{table}

Since the GPU is important in this type of applications, we show the specs of the GPU used in Table \ref{tab:gpu-specs}.

\begin{table}[h!]
\centering
\begin{tabular}{ll}
\multicolumn{2}{c}{GPU specification} \\
Adapter type   & NVIDIA GeForce GTX 1080 Ti \\
Chip Type  &  GeForce GTX 1080 Ti \\
DAC Type & Integrated RAMDAC \\
Available memory & 45025 MB
\end{tabular}
\caption{GPU specification.}
\label{tab:gpu-specs}
\end{table}

As for the VR headset hardware, we used a Oculus Quest. In Table \ref{tab:oculus-specs} we can see the hardware specifications for this type of headset.

\begin{table}[h!]
\centering
\begin{tabular}{ll}
\multicolumn{2}{c}{Oculus Quest specifications} \\
Panel Type   & Dual OLED 1600x1440 \\
Supported Refresh Rate  &  72Hz \\
Tracking & Inside out, 6DOF \\
CPU & Qualcomm® Snapdragon 835 \\
GPU & Qualcomm® Adreno™ 540 GPU \\
Memory & 4GB total
\end{tabular}
\caption{Oculus Quest specifications.}
\label{tab:oculus-specs}
\end{table}

The experiments were run in Unity, the same software that was used for the development. The version of Unity used for the expriments is 2018.4.10f1. Also Vulkan is used as the graphic API.

% @TODO Write about how scalability was tested.
A test scene was built in order to test the network and the scalability of it. A random network is built and we test how comfortable it is to navigate through it. This is measured by the FPS rate.

VR profiling is a technique used to get an overview of the performance of our application. This is usually done in order to find bottlenecks so that we can eliminate them and improve the application's performance.

To profile the application we used the built-in profiler in Unity, the software used for the development. The Unity Profiler gives information about per-fram CPU and GPU performance metrics.

\section{Scalability of the network}
Creating the lines for relationships.

\section{Questionnaire to evaluate the system}
One of the questions that we asked ourselves during the evaluation process was about the comfortability of using BigNext VR to explore a biological network. This is an important aspect when building VR applications. Some of the aspects to take into account are for instance the motion sickness or the intuitiveness. In order to evaluate this we made a questionnaire for bioinformaticians that would test the application. Unfortunately, due to the Covid-19 situtation\cite{covid_19}, we were not able to carry out the questionnaire with people. The reason is because it wasn't possible to test BigNet VR with people on a single Oculus Quest device without avoiding the social distancing rules.
We estimated to have around 10 participants with knowdledge in bioinformatics to test the application. With this number of participants we could have made some statistics and obtained feedback for future improvement.

The following questionnaire is divided in four sections; a general section about VR headsets, a section about comfortability exploring the network using BigNet VR, a section about the different actions in BigNet VR and finally a section about feedback.

To complete the questionnaire, the teste has to indicate the level of agreement or disagreement with each of the  statements, mark yes or no when it is asked and in the feedback section reply the questions with constructive feedback if possible.\\

Questionnaire section 1: VR headsets.
\begin{enumerate}
  \item Have you ever used a VR headset before?\\
  Yes / No

  \item Have you ever used a Oculus Quest headset before?\\
  Yes / No

  \item I feel comfortable using a VR headset.\\
  Strongly agree / Agree / Neutral / Disagree / Strongly Disagree

  \item I feel comfortable using the Oculus Quest headset.\\
  Strongly agree / Agree / Neutral / Disagree / Strongly Disagree\\
\end{enumerate}

Questionnaire section 2: Comfortability exploring a biological network with BigNet VR.
\begin{enumerate}
  \item I feel comfortable moving around the virtual environment using the teleport functionality.\\
  Strongly agree / Agree / Neutral / Disagree / Strongly Disagree

  \item I feel comfortable rotating to any direction.\\
  Strongly agree / Agree / Neutral / Disagree / Strongly Disagree

  \item I feel comfortable visualizing the network by moving my head.\\
  Strongly agree / Agree / Neutral / Disagree / Strongly Disagree

  \item I feel comfortable selecting the nodes to visualize the relationships.\\
  Strongly agree / Agree / Neutral / Disagree / Strongly Disagree

  \item I feel comfortable moving the network to the position that I want.\\
  Strongly agree / Agree / Neutral / Disagree / Strongly Disagree

  \item I feel comfortable scaling the network.\\
  Strongly agree / Agree / Neutral / Disagree / Strongly Disagree

  \item I feel comfortable using the UI menu to filter the data from the network.\\
  Strongly agree / Agree / Neutral / Disagree / Strongly Disagree\\
\end{enumerate}

Questionnaire section 3: Performing different actions in BigNet VR to explore the biological network.
\begin{enumerate}
  \item It is intuitive to manipulate the network using the controllers.\\
  Strongly agree / Agree / Neutral / Disagree / Strongly Disagree

  \item The different actions in the controllers are easy to learn and remember.\\
  Strongly agree / Agree / Neutral / Disagree / Strongly Disagree

  \item I can move the network to any position that I want.\\
  Strongly agree / Agree / Neutral / Disagree / Strongly Disagree

  \item I can scale the network to any size that I want.\\
  Strongly agree / Agree / Neutral / Disagree / Strongly Disagree

  \item I can select any node that I want.\\
  Strongly agree / Agree / Neutral / Disagree / Strongly Disagree

  \item I can easily visualize the relationships of any node.\\
  Strongly agree / Agree / Neutral / Disagree / Strongly Disagree

  \item I can easily filter the data by using the UI menu.\\
  Strongly agree / Agree / Neutral / Disagree / Strongly Disagree\\
\end{enumerate}

Questionnaire section 4: Feedback.
\begin{enumerate}
  \item Did you experience any difficulties exploring the biological network? If so, indicate which ones.\\
  Yes / No

  \item Is there anything that could be improved for the visualization of biological data in BigNet VR? If so, write your suggestions.\\
  Yes / No

  \item Write any feedback and comments that you have about the exploration of biological networks with BigNet VR.
\end{enumerate}


\chapter{Related work}

\section{BioVR}

\section{CellexaVR}

\section{BigTop}


\chapter{Conclusion}
We have developed GeneNet VR, a Virtual Reality application for the visualization of large biological networks. We used two datasets from MIxT, a real application with visualization problems, as a case study and solved scalability and information overload problems. GeneNet VR gives bioinformaticians the possibility to explore large biological networks for pattern finding using the Oculus Quest, a standalone cheap VR headset.

Previous work for the visualization of large biological networks has shown common problems like the hairball problem, cumbersome interactions and scalability issues. We solve these in GeneNet VR by taking advantage of the rich interactivity that VR offers in order to balance out the amount of information. We have implemented several interaction and visualization techniques that help the users explore the large datasets. A locomotion system is used, providing the user a way to move around in the scene, solving also object occlusion problems common in 3D spaces. The user can also move the network around, zoom in it and filter the nodes using a 2-dimensional UI. The edges of the network are shown for each node when the user selects them using a laser pointer. Also, a novel feature allows the users compare two datasets in real time. We demonstrate as well that the networks can be explored in cheap VR hardware by running a performance evaluation, which makes GeneNet VR light to use.

We ran several performance experiments for our case study and demonstrated that GeneNet VR has a good performance when exploring the networks from MIxT. We also evaluated several interactions that are commonly used during the visualization process and concluded that they achieved high FPS resuts. We learned from the experiments that the average frame time oscilates between 6.4 and 8.7 milliseconds, which is much lower than the limit from the Oculus guidelines (13.9 ms). We also compared the performance between the PC and the Oculus Quest hardware and the results indicated that the Oculus Quest is around 30\% slower, but it is still inside the required FPS.

Through several interviews with biologists and computer scientists from UiT, we got positive feedback, standing out that GeneNet VR is an interesting and usfeul tool to explore biological networks. The system is also easy to learn for the interviewees and the interactivity is smooth, making it easy to explore the networks and with potential to find novel patterns in them. We also made a list with further requirements that we extracted from the interviews that will help us improve GeneNet VR in the future.

We believe that GeneNet VR is an important and useful tool for bioinformaticians. We have traced out the guidelines that enable VR as an affordable and advantageous option for the visualization of large biological networks. The rich interactivity that VR offers and the advancement in hardware has allowed us to solve visualization and scalability problems that are common in similar visualization tools. Now we ask ourselves, what are the next steps of GeneNet VR? And also, can we use a similar solution to visualize similar networks from other fields? Thanks to the interviews that we carried out, we think that GeneNet VR has potential for future improvements and can definitely be useful in other fields, for example for the visualization of drug networks and social networks.


\chapter{Future work}
Future work
\section{Development}
List of issues in \ref{fig:issues}.
\begin{figure}[h!]
    \setlength{\tempheight}{15ex}
    \centering
    \includegraphics[width=\textwidth]{issues_github}
    \caption{List of issues from Github. They are tagged with "bug" or "enhancement" to specify what kind of issue it is about.}
    \label{fig:issues}
\end{figure}

\section{Evaluation}


\printbibliography

% \begin{appendices}
% \chapter{Appendix A}
% \begin{figure}[h!]
  \centering
  \begin{tikzpicture}
    \begin{axis}
    [ xlabel=Number of edges,
    ylabel=Number of nodes,
    xmin=0,xmax=4300,
    ymin=0,ymax=70,
    ]
    \addplot[scatter, only marks] file{data/biopsyEdges.dat};
    \end{axis}

    \end{tikzpicture}
\caption{Scatter plot showing a distribution of the number of edges in the biopsy dataset.}
\label{fig:edges_nodes_biopsy}
\end{figure}

% \end{appendices}

\backmatter

\end{document}
